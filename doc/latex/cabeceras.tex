%
% cabeceras.tex
%
% Copyright 2003, Diego Berrueta Muñoz
%
% Cabeceras comunes
%

\usepackage[T1]{fontenc}

%Glosario

\usepackage[toc,style=treenoname,order=word,counter=section]{glossaries}

\usepackage{xspace}


\usepackage{tikz,times}

% cambia algunas fuentes (utilidad dudosa)
\usepackage[scaled=0.92]{helvet}
\usepackage{pifont}
\usepackage{courier}

% cambia algunas fuentes en modo matemático a Palatino
\usepackage{mathpazo}

% españolización
\usepackage[spanish]{babel}
\usepackage[utf8]{inputenc}
%\extrasspanish

% gráficos y colores
\usepackage{rotating}
\usepackage{graphicx}
\usepackage{color}
%\usepackage[all]{xy}
%\usepackage{pstricks}
\usepackage{pst-node}
%\usepackage[dvips,usenames]{pstcol}
%\usepackage{pdftricks}
%\usepackage{pst-uml}  % para hacer diagramas UML
%\usepackage{rail}     % para hacer diagramas de gramáticas

% mejoras visuales
\usepackage{enumerate}
\usepackage{fancyhdr}  % para configurar los encabezados
\usepackage{fancybox}  % para hacer cajitas
\usepackage[normal,oneline,sf,bf]{caption2}
\usepackage{titlesec}  % para configurar los títulos de sección
\usepackage{paralist}


\usepackage{epigraph}

% citas, referencias e índices
\usepackage{cite}
%\usepackage{citesort}   % da errores al compilar
\usepackage{makeidx}

% incrustaciones de código fuente
%\usepackage[norules,nolineno]{lgrind}
\usepackage{verbatim}
\usepackage{listings}
%\usepackage{noweb,a4wide}


%\usepackage{textcomp}
\usepackage[right]{eurosym}

% columnas
\usepackage{multicol}

% tablas
\usepackage{longtable}
%\usepackage{ltxtable}

% impresión elegante de URLs
\usepackage{url}

\makeatletter
\def\url@pfcstyle{%
  \@ifundefined{selectfont}{\def\UrlFont{\sf}}{\def\UrlFont{\small\ttfamily}}}
\makeatother
%% Now actually use the newly defined style.
\urlstyle{pfc}


% márgenes
\usepackage[a4paper, left=30mm, right=20mm, top=25mm, bottom=25mm]{geometry}
%\usepackage[a4paper, left=20mm, right=20mm, top=25mm, bottom=25mm]{geometry}

% 
% \usepackage[a4,center,cam]{crop}
\usepackage{blindtext}

% salida en PDF navegable
%\usepackage{hyperref}
\usepackage[plainpages=false,colorlinks, linkcolor=black]{hyperref}

% quitar en versión final
%\usepackage{showkeys}   % depuración de etiquetas y referencias
%\usepackage{showidx}    % depuración de índice

% configuración del paquete "listings"

\lstset{%
    language=Java,
	basicstyle=\footnotesize\sffamily,
	keywordstyle=\bfseries, %\color{darkred}
 	stringstyle=\itshape, %\color{violet}
 	commentstyle=\itshape, %\color{blue}
 	showspaces=false,
 	showtabs=false,
 	showstringspaces=false,
 	frame=trBL,
        frameround=tttt,
        %backgroundcolor=\color{lightyellow},
	inputencoding=utf8,
 	extendedchars=true,
 	numbers=none,
        aboveskip=0.5cm,
        belowskip=0.5cm,
        xleftmargin=1cm,
        xrightmargin=1cm,
	breaklines=true
}
\definecolor{darkred}{rgb}{0.5, 0, 0}
\definecolor{violet}{rgb}{1, 0, 1}
\definecolor{lightyellow}{rgb}{1,1,0.8}


%%%%%%%%%%%%%%%%%%%%%%%%%%%%%%%%%%%%%%%%%%%%%%%%%%%%%%%%%%%%%%%%%%%%%%
% cabeceras y pies de página (con el paquete "fancyhdr")
\headheight 15pt
%\addtolength{\headwidth}{\marginparsep}
%\addtolength{\headwidth}{\marginparwidth}
%\renewcommand{\chaptermark}[1]{\markboth{\MakeUppercase{#1}}{}}
%\renewcommand{\sectionmark}[1]{\markright{\thesection\ #1}}
%\fancyhead[LE,RO]{\textbf{\thepage}}
%\fancyhead[RE]{\textit{\leftmark}}
%\fancyhead[LO]{\rightmark}
%\fancyfoot[LCR]{}
\fancyhead{} % Todos los campos en blanco en la cabecera
\fancyfoot{} % Lo mismo al pie
\fancyhead[RO, LE]{\thepage}
\fancyhead[LO, RE]{\slshape\leftmark}
\renewcommand{\headrulewidth}{0.5pt}
\renewcommand{\footrulewidth}{0.5pt}


%%%%%%%%%%%%%%%%%%%%%%%%%%%%%%%%%%%%%%%%%%%%%%%%%%%%%%%%%%%%%%%%%%%%%%
% títulos de secciones (con el paquete "titlesec")
\titleformat{\chapter}[display]
	{\fontfamily{pag}\selectfont\Huge}
	{\LARGE\chaptertitlename\ \thechapter}{20pt}{\bfseries}
\titleformat{\section}
	{\fontfamily{phv}\selectfont\LARGE}
	{\thesection}{1em}{\bfseries}[\titlerule]
\titleformat{\subsection}
	{\fontfamily{phv}\selectfont\Large}
	{\thesubsection}{1em}{\bfseries}
\titleformat{\subsubsection}
	{\fontfamily{phv}\selectfont\large}
	{\thesubsubsection}{1em}{\bfseries}


%%%%%%%%%%%%%%%%%%%%%%%%%%%%%%%%%%%%%%%%%%%%%%%%%%%%%%%%%%%%%%%%%%%%%%
% espaciado entre párrafos
\addtolength{\parskip}{+0.2cm}


%%%%%%%%%%%%%%%%%%%%%%%%%%%%%%%%%%%%%%%%%%%%%%%%%%%%%%%%%%%%%%%%%%%%%%
% salida en PDF navegable (con el paquete "hyperref")
\hypersetup{bookmarks,
	bookmarksnumbered,
%	colorlinks, % quitar en las versiones impresas
	hyperindex,
	%linkcolor=red,
	%anchorcolor=black,
	%citecolor=green,
	citecolor=violet,
	%filecolor=magenta,
	%menucolor=red,
	%pagecolor=red,
	%urlcolor=cyan,
	pdftitle={Ejercicios Básicos en Java-Jose María Alvarez Rodríguez},
	pdfauthor={Jose María Alvarez Rodríguez}
	pdffitwindow,
	plainpages=false,
	pageanchor=false,
	pdfstartview={}}


%%%%%%%%%%%%%%%%%%%%%%%%%%%%%%%%%%%%%%%%%%%%%%%%%%%%%%%%%%%%%%%%%%%%%%
% profundidad de secciones y numeración
%\setcounter{tocdepth}{4}
\setcounter{secnumdepth}{3}


%%%%%%%%%%%%%%%%%%%%%%%%%%%%%%%%%%%%%%%%%%%%%%%%%%%%%%%%%%%%%%%%%%%%%%
% división silábica
\hyphenation{pu-bli-ca-ción}   

%%Tablas
\addto\captionsspanish{
        \def\listtablename{\'Indice de tablas}%
        \def\tablename{Tabla}} 

%%%Math
\usepackage{latexsym}
\usepackage{amsmath}
\usepackage{amssymb}
\usepackage{amsthm}

\usepackage{algorithm}
\usepackage{algorithmic}
\usepackage{multirow}
\usepackage{rotating}


\newtheorem{theorem}{Theorem}[section]
\newtheorem{proposition}[theorem]{Proposición}
\newtheorem{lemma}[theorem]{Lema}
\newtheorem{definition}[theorem]{Definición}
\newtheorem{examples}[theorem]{Ejemplos}
\newtheorem{remarks}[theorem]{Remarks}
\newtheorem{corollary}[theorem]{Corolario}
\newtheorem{remark}[theorem]{Remark}
\newtheorem{example}[theorem]{Ejemplo}
\newtheorem{conjecture}[theorem]{Conjecture}
\newtheorem{note}[theorem]{Nota}



\newsavebox\FrameBox
\newenvironment{Frame}{%
  \par\setbox\FrameBox\hbox\bgroup\minipage{0.9\textwidth}\parskip\baselineskip\ignorespaces
}{%
  \endminipage\egroup\fbox{\box\FrameBox}\par
}

\newcommand{\si}{$\oplus$\xspace}
\newcommand{\no}{$\ominus$\xspace}
\newcommand{\na}{$\odot$\xspace}


%Extraer
\newcommand{\linkeddata}{\textit{Linked Data}\xspace}
\newcommand{\opendata}{\textit{Open Data}\xspace}
\newcommand{\lod}{\textit{Linking Open Data}\xspace}
\newcommand{\ogd}{\textit{Open Government Data}\xspace}
\newcommand{\datasets}{\textit{datasets}\xspace}
\newcommand{\dataset}{\textit{dataset}\xspace}
\newcommand{\provenance}{\textit{provenance}\xspace}
\newcommand{\trust}{\textit{trust}\xspace}
\newcommand{\egov}{\textit{e-government}\xspace}
\newcommand{\pusi}{\textit{Public Sector Information}\xspace}
\newcommand{\gd}{\textit{Government Data}\xspace}
\newcommand{\wod}{Web de Datos\xspace}
\newcommand{\wode}{\textit{Web of Data}\xspace}
\newcommand{\eproc}{\textit{\gls{e-Procurement}}\xspace}
\newcommand{\gld}{\textit{Government Linked Data}\xspace}
