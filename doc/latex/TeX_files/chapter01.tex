\chapter{Ejemplos básicos en Java}

A continuación se dispone de una serie de ejemplos básicos en Java (ordenados alfabéticamente) para trabajar con:

\begin{itemize}
	\item Declaración de variables.
	\item Operaciones simples.
\end{itemize}

\centering

\begin{framed}
		\lstinputlisting[label=samplecode,caption=Detectar si un año es bisiesto.]{ ../../examples/src/main/java/es/uc3m/programacion/basico/AnyoBisiesto.java}
\end{framed}

\begin{framed}
	\lstinputlisting[label=samplecode,caption=Convertir coordenadas cartesianas a polares. ]{ ../../examples/src/main/java/es/uc3m/programacion/basico/CartesianoAPolar.java}
\end{framed}

\begin{framed}
	\lstinputlisting[label=samplecode,caption=Calcular el número de día de la semana dada una fecha. ]{ ../../examples/src/main/java/es/uc3m/programacion/basico/DiaDeLaSemana.java}
\end{framed}

\begin{framed}
	\lstinputlisting[label=samplecode,caption=Calcular la distancia euclídea entre dos puntos. ]{ ../../examples/src/main/java/es/uc3m/programacion/basico/DistanciaEuclidea.java}
\end{framed}

\begin{framed}
	\lstinputlisting[label=samplecode,caption=Ejemplos de uso de Math API. ]{ ../../examples/src/main/java/es/uc3m/programacion/basico/EjemplosMathAPI.java}
\end{framed}


\begin{framed}
	\lstinputlisting[label=samplecode,caption=Converstir grados Celsius a Fahrenheit. ]{ ../../examples/src/main/java/es/uc3m/programacion/basico/FahrenheitACelsius.java}
\end{framed}


\begin{framed}
	\lstinputlisting[label=samplecode,caption=Intercambiar el valor de dos variables. ]{ ../../examples/src/main/java/es/uc3m/programacion/basico/IntercambiarVariables.java}
\end{framed}

\begin{framed}
	\lstinputlisting[label=samplecode,caption=Calcular el valor mayor de dos números reales. ]{ ../../examples/src/main/java/es/uc3m/programacion/basico/MayorDeDosReales.java}
\end{framed}


\begin{framed}
	\lstinputlisting[label=samplecode,caption=Calcular el valor mayor de tres números enteros. ]{ ../../examples/src/main/java/es/uc3m/programacion/basico/MayorDeTresEnteros.java}
\end{framed}



\begin{framed}
	\lstinputlisting[label=samplecode,caption=Calcular el valor mayor de tres números reales. ]{ ../../examples/src/main/java/es/uc3m/programacion/basico/MayorDeTresReales.java}
\end{framed}


\begin{framed}
	\lstinputlisting[label=samplecode,caption=Calcular la media de dos números. ]{ ../../examples/src/main/java/es/uc3m/programacion/basico/MediaDosNumeros.java}
\end{framed}


\begin{framed}
	\lstinputlisting[label=samplecode,caption=Calcular la sensación térmica. ]{ ../../examples/src/main/java/es/uc3m/programacion/basico/SensacionTermica.java}
\end{framed}


\begin{framed}
	\lstinputlisting[label=samplecode,caption=Calcular la suma de dos números. ]{ ../../examples/src/main/java/es/uc3m/programacion/basico/SumaDosNumeros.java}
\end{framed}




