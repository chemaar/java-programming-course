\chapter{Ejemplos de sentencias condicionales y bucles en Java}

A continuación se dispone de una serie de ejemplos básicos en Java (ordenados alfabéticamente) para trabajar con:

\begin{itemize}
	\item Sentencias condicionales simples.
	\item Bucles simples.
\end{itemize}

\centering

\begin{framed}
	\lstinputlisting[label=samplecode,caption=Contar los divisores de un número. ]{ ../../examples/src/main/java/es/uc3m/programacion/bucles/ContarDivisores.java}
\end{framed}

\begin{framed}
	\lstinputlisting[label=samplecode,caption=Contar números impares hasta un número determinado. ]{ ../../examples/src/main/java/es/uc3m/programacion/bucles/ContarNumerosImpares.java}
\end{framed}

\begin{framed}
	\lstinputlisting[label=samplecode,caption=Calcular el cubo de los primeros veinte números naturales utilizando Math API de Java. ]{ ../../examples/src/main/java/es/uc3m/programacion/bucles/CuboNumerosMathAPI.java}
\end{framed}

\begin{framed}
	\lstinputlisting[label=samplecode,caption=Calcular el factorial de un número. ]{ ../../examples/src/main/java/es/uc3m/programacion/bucles/Factorial.java}
\end{framed}

\begin{framed}
	\lstinputlisting[label=samplecode,caption=Mostrar la secuencia de Fibonacci. ]{ ../../examples/src/main/java/es/uc3m/programacion/bucles/Fibonacci.java}
\end{framed}

\begin{framed}
	\lstinputlisting[label=samplecode,caption=Calcular la potencia de $a^b$. ]{ ../../examples/src/main/java/es/uc3m/programacion/bucles/MiPotencia.java}
\end{framed}


\begin{framed}
	\lstinputlisting[label=samplecode,caption=Mostrar los números pares hasta un número determinado utilizando diferentes tipos de bucles. ]{ ../../examples/src/main/java/es/uc3m/programacion/bucles/MostrarNumerosPares.java}
\end{framed}


\begin{framed}
	\lstinputlisting[label=samplecode,caption=Mostrar los números naturales hasta un número determinado utilizando diferentes tipos de bucles. ]{ ../../examples/src/main/java/es/uc3m/programacion/bucles/MostrarVeinteNaturales.java}
\end{framed}


\begin{framed}
	\lstinputlisting[label=samplecode,caption=Mostrar los números naturales de forma descendente utilizando diferentes tipos de bucles. ]{ ../../examples/src/main/java/es/uc3m/programacion/bucles/MostrarVeinteNaturalesDescendente.java}
\end{framed}


\begin{framed}
	\lstinputlisting[label=samplecode,caption=Detectar si un número natural es un número de Armstrong. ]{ ../../examples/src/main/java/es/uc3m/programacion/bucles/NumeroArmstrong.java}
\end{framed}


\begin{framed}
	\lstinputlisting[label=samplecode,caption=Calcular el número combinatorio. ]{ ../../examples/src/main/java/es/uc3m/programacion/bucles/NumeroCombinatorio.java}
\end{framed}



\begin{framed}
	\lstinputlisting[label=samplecode,caption=Detectar si un número natural es un palíndromo. ]{ ../../examples/src/main/java/es/uc3m/programacion/bucles/NumeroPalindromo.java}
\end{framed}


\begin{framed}
	\lstinputlisting[label=samplecode,caption=Detectar si un número natural es número perfecto. ]{ ../../examples/src/main/java/es/uc3m/programacion/bucles/NumeroPerfecto.java}
\end{framed}


\begin{framed}
	\lstinputlisting[label=samplecode,caption=Detectar si un número natural es número primo. ]{ ../../examples/src/main/java/es/uc3m/programacion/bucles/NumeroPrimo.java}
\end{framed}



\begin{framed}
	\lstinputlisting[label=samplecode,caption=Sumar $n$ números enteros. ]{ ../../examples/src/main/java/es/uc3m/programacion/bucles/SumarNEnteros.java}
\end{framed}


\begin{framed}
	\lstinputlisting[label=samplecode,caption=Sumar los primeros 20 $n$ números naturales. ]{ ../../examples/src/main/java/es/uc3m/programacion/bucles/SumarVeinteNaturales.java}
\end{framed}



